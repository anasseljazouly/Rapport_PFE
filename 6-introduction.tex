\chapter*{Introduction générale}
\addcontentsline{toc}{chapter}{Introduction générale}
\markboth{Introduction générale}{Introduction générale}
\label{chap:introduction}
Dans le cadre du projet \textbf{SG CONNECT}, nous nous concentrons sur l'enrichissement des fonctionnalités de l'application banque à distance web et mobile. Ce projet est divisé en deux parties distinctes : la partie UNIBANK, chargée par le backend, et la partie BANKUP, chargée par le front end. Ces deux équipes travaillent en collaboration pour créer une expérience utilisateur complète.\\

La partie \textbf{UNIBANK}, similaire à la préparation de la pâte à pizza, se concentre sur les aspects techniques et sécuritaires de l'application. Leur rôle principal est de mettre en place une infrastructure solide pour gérer les transactions bancaires, sécuriser les communications avec les systèmes externes et assurer la confidentialité des données. Ils créent ainsi la base fondamentale sur laquelle repose l'application SG CONNECT.\\

D'autre part, l'équipe \textbf{BANKUP}, telle celle qui ajoute la garniture sur la pizza, se concentre sur l'expérience utilisateur et l'interface utilisateur de l'application. Leur responsabilité est de concevoir et d'implémenter des fonctionnalités conviviales qui permettent aux clients d'interagir facilement avec l'application. Ils veillent à ce que l'application offre une expérience fluide, agréable et adaptée aux besoins des utilisateurs.\\

La \textbf{Digital Factory} de la SGABS joue un rôle clé dans ce projet, en tant qu'accélérateur de la transformation digitale pour les filiales de la région AFS. En respectant les réglementations sécuritaires du groupe SG, l'équipe UNIBANK sécurise les transactions bancaires et les communications avec les systèmes externes, en utilisant la plateforme digitale UNIBANK. Cette plateforme, composée de différentes couches telles que la couche Gateway et les couches APIs métier et APIs Bridge, résout les problèmes liés à l'ouverture des systèmes bancaires et apporte une valeur métier et technique aux projets de la SG.\\

Pendant mon stage chez SG ABS, j'ai rejoint l'équipe BANKUP et j'ai contribué au développement de nouvelles fonctionnalités pour l'application SG CONNECT dédiée aux clients de la zone AFS. En tant qu'analyste métier, j'ai pris en charge la conception fonctionnelle en exprimant les besoins des utilisateurs et en rédigeant des user stories assignées aux développeurs. J'ai également effectué des tests fonctionnels pour garantir que nos fonctionnalités répondaient aux exigences identifiées. En tant que développeur, j'ai utilisé des outils techniques tels que React JS, React Native, Cypress et Git pour implémenter les fonctionnalités définies, en respectant les normes et les standards de la Digital Factory.\\

Ce rapport présente une synthèse complète du travail effectué dans le cadre de ce projet. Le premier chapitre offre un aperçu général du projet SG CONNECT, suivi du deuxième chapitre qui détaille l'analyse et l'étude fonctionnelle. Le troisième chapitre se concentre sur l'étude conceptuelle du projet, tandis que le dernier chapitre présente la réalisation du travail effectué par les équipes UNIBANK et BANKUP.\\