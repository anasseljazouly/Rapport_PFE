\chapter{Realisation }
Dans ce chapitre, nous énumérerons les plus grandes difficultés rencontrées dans le processus de réalisation du projet. Par conséquent, nous expliquerons le rôle des fonctions apparaissant dans le code source.


\section{Difficultes rencontrees}


Cette section détaille le processus d'exécution du projet et les problèmes qui en découlent.
\subsection{Le temps}
Le temps était la contrainte la plus dur à gérer vu qu'on vient de s'initier à la réalisation du projet et que c'est notre premiere experience professionnelle , et le délai était insuffisant voire même trop serré pour perfectionner notre projet, sa réalisation dans cette durée était un vrai challenge.


\subsection{Le travail en equipe}
L'un des challenge les plus dure etait de travailler en equipe et cooperer entre nous, ainsi que comprendre le code ecrit par mes colegues.


\section{Outils}
\subsection{Mac OS}
Mac OS est un système d’exploitation partiellement propriétaire développé et commercialisé par Apple depuis 1998, dont la version la plus récente est macOS Big Sur (version 11)5 lancée le 12 novembre 2020. Avec iOS, iPadOS, watchOS et tvOS, il fait partie des systèmes d'exploitation d'Apple.
\begin{figure}[!h]
\begin{center}
\includegraphics[width=6cm]{presentation image/MacOS.png}
\end{center}
%légende de l'image
\caption{Logo MacOS}
\end{figure}

\subsection{WildFly}
WildFly, anciennement JBoss Application Server ou JBoss, est un serveur d'applications Java EE Libre écrit en Java, publié sous licence GNU LGPL. Étant écrit en Java, WildFly peut être utilisé sur tout système d'exploitation fournissant une machine virtuelle Java (JVM). Le nom JBoss est aujourd'hui utilisé pour JBoss EAP, produit dérivé WildFly et faisant l'objet d'un support commercial3.
\begin{figure}[!h]
\begin{center}
\includegraphics[width=6cm]{presentation image/wildfly.png}
\end{center}
%légende de l'image
\caption{logo WildFly}
\end{figure}

\subsection{PhpMyAdmin}
PhpMyAdmin est une application Web de gestion pour les systèmes de gestion de base de données MySQL et MariaDB, réalisée principalement en PHP et distribuée sous licence GNU GPL. C’est un outil qui va permettre de gérer l’administration de la base de données et de visualiser rapidement l'état de notre base de données et de la modifier, sans avoir à écrire de requêtes SQL.
\begin{figure}[!h]
\begin{center}
\includegraphics[width=6cm]{presentation image/phpmyadmin.png}
\end{center}
%légende de l'image
\caption{logo phpMyAdmin}
\end{figure}

\subsection{Xcode}
Xcode est un environnement de développement pour macOS, ainsi que pour iOS, watchOS et tvOS. L'API Cocoa permet de programmer avec les langages suivants :  Objective-C  Ruby  Swift (nouveau langage de programmation d'Apple présenté à la WWDC 2014). Fourni avec toute une suite logicielle (graphiques, audio, etc.) pour développeurs et programmeurs, il permet de créer des logiciels utilisant toutes les fonctionnalités de macOS et d'UNIX. Cet environnement peut être obtenu gratuitement sur le Mac App Store. Il était fourni en standard avec chaque Mac, sur les disques d'installation de Mac OS X 10.6, Mac OS X 10.5, 10.4 et 10.3. Il n'est pas pré-installé sur l'ordinateur et doit être installé séparément. natives.
\begin{figure}[!h]
\begin{center}
\includegraphics[width=4cm]{presentation image/Xcode.png}
\end{center}
%légende de l'image
\caption{logo Xcode}
\end{figure}

\subsection{COCOPODS}
CocoaPods est un gestionnaire de dépendances au niveau de l'application pour Objective-C, Swift et tout autre langage qui s'exécute sur le runtime Objective-C, tel que RubyMotion, qui fournit un format standard pour la gestion des bibliothèques externes.
\begin{figure}[!h]
\begin{center}
\includegraphics[width=6cm]{presentation image/cocoapods.png}
\end{center}
%légende de l'image
\caption{logo COCOPODS}
\end{figure}


\subsection{Maven}
Apache Maven est un outil de gestion et d'automatisation de production des projets logiciels Java en général et Java EE en particulier.il permet une gestion efficace des dépendances.
Maven est géré par l'organisation Apache Software Fondation. Les informations minimales d’un projet maven :\\

• GroupId : le nom de l’organisation qui gères le projet.\\

• ArtifactId : identifiant de projet.\\

• Version : la version de projet.\\

• Packaging : format du package à générer (jar, war…).\\

Le build lifecycle maven :\\

• Validate : vérifie que la configuration projet est correcte (POM, pas d'éléments manquants...).\\

• Compile : compile les sources du projet.\\

• Test : teste le code compilé avec les classes de tests unitaires contenues dans le projet.\\

• Package : package les éléments issus de la compilation dans un format distribuable (JAR, WAR...).\\

• Install : installe le package dans votre repository local.\\

• Deploy : envoie le package dans le repository distant défini dans le POM.\\

\begin{figure}[!h]
\begin{center}
\includegraphics[width=6cm]{presentation image/maven.png}
\end{center}
%légende de l'image
\caption{logo Maven}
\end{figure}

\subsection{Android Studio}
Android Studio est un environnement de développement pour développer des applications mobiles Android. Il est basé sur IntelliJ IDEA et utilise le moteur de production Gradle. Il peut être téléchargé sous les systèmes d'exploitation Windows, macOS, Chrome OS et Linux.
\newpage

\begin{figure}[!h]
\begin{center}
\includegraphics[width=6cm]{presentation image/android.png}
\end{center}
%légende de l'image
\caption{logo Android Studio}
\end{figure}

\subsection{Intellij IDEA}
Intellij IDEA est un environnement de développement intégré de technologie Java destiné au développement de logiciels informatiques.
Chaque aspect d'Intellij IDEA a été spécifiquement conçu pour optimiser la productivité des développeurs. Combinées, l'assistance au codage et la conception ergonomique permettent un développement non seulement productif, mais également agréable. Il intègre aussi des outils de développement libre tel que Git, Maven qu’on utilise.
\begin{figure}[!h]
\begin{center}
\includegraphics[width=4cm]{presentation image/intelj.png}
\end{center}
%légende de l'image
\caption{logo Intelj IDEA}
\end{figure}

\section{Technologies}
\subsection{Mysql}
MySQL est un système de gestion de bases de données relationnelles (SGBDR). Il est distribué sous une double licence GPL et propriétaire. Il fait partie des logiciels de gestion de base de données les plus utilisés au monde, autant par le grand public (applications web principalement) que par des professionnels, en concurrence avec Oracle, PostgreSQL et Microsoft SQL Server.\\

MySQL est un serveur de bases de données relationnelles SQL développé dans un souci de performances élevées en lecture, ce qui signifie qu'il est davantage orienté vers le service de données déjà en place que vers celui de mises à jour fréquentes et fortement sécurisées. Il est multithread et multi-utilisateur.\\

MySQL fait partie des exigences du responsable du projet afin de pouvoir intégrer l’application dans le système lors du déploiement.
\newpage
\begin{figure}[!h]
\begin{center}
\includegraphics[width=6cm]{presentation image/mysql.png}
\end{center}
%légende de l'image
\caption{Logo MySQL}
\end{figure}

\subsection{Spring boot}
Spring Boot est un Framework de développement JAVA open source pour construire et définir l’infrastructure d’une application JAVA. Spring Boot est une version de Spring qui permet de simplifier à l’extrême, Entre autres :
La gestion des dépendances Maven, Auto-Configuration.
Spring Framework est le framework de développement d'applications le plus populaire de Java. La principale caractéristique de Spring Framework est l’injection de dépendance ou l’inversion de contrôle (IoC).
Spring Boot et notamment JAVA sont aussi une exigence du responsable du projet é tant donné que le langage de programmation du système est en JAVA.
\begin{figure}[!h]
\begin{center}
\includegraphics[width=6cm]{presentation image/springboot.png}
\end{center}
%légende de l'image
\caption{logo Spring boot}
\end{figure}

\subsection{Git}
Git est le logiciel de contrôle de version le plus utilisé, il permet de garder une trace sur ce qui a été fait et de revenir à une phase précédente dans le cas où on décide de revenir sur des changements. Il facilite aussi la résolution des erreurs et permet de noter les modifications apportées à chaque version pour faciliter le travail en équipe.
On utilise le stockage cloud de Github pour faciliter la collaboration entre les membres de l’équipe et travailler sur le même projet en même temps.
\begin{figure}[!h]
\begin{center}
\includegraphics[width=4cm]{presentation image/git.png}
\end{center}
%légende de l'image
\caption{logo Spring boot}
\end{figure}

\subsection{Flutter}
Flutter est un Framework de développement d’applications multiplateforme, conçu par Google, dont la première version a été publiée sous forme de projet open source à la fin de l’année 2018. Flutter met à disposition une grande variété de bibliothèques d’éléments d’IU standard pour Android et iOS. Flutter est principalement utilisé pour le développement d'applications Android et iOS, sans nécessiter la création d’une base de code propre à chacun de ces deux systèmes si différents l’un de l’autre. De cette manière, les applications de smartphone se comportent, sur les différents appareils, comme de réelles applications natives.\\

Les widgets sont le composant principal de toute application flutter. Il agit comme une interface utilisateur pour l’utilisateur d’interagir avec l’application. Toute application flutter est elle-même un widget qui se compose d’une combinaison de widgets. Dans une application standard, la racine définit la structure de l’application suivie d’un widget MaterialApp qui maintient essentiellement ses composants internes en place. C’est là que les propriétés de l’interface utilisateur et de l’application elle-même sont définies. Le MaterialApp a un widget Scaffold qui se compose des composants visibles (widgets) de l’application. L’échafaudage a deux propriétés primaires à savoir le corps et l’appbar. Il contient tous les widgets enfants et c’est là que toutes ses propriétés sont définies.

\begin{figure}[!h]
\begin{center}
\includegraphics[width=4cm]{presentation image/Flutter.png}
\end{center}
%légende de l'image
\caption{logo Flutter}
\end{figure}

\subsection{Firebase}
Firebase est un ensemble de services d'hébergement pour n'importe quel type d'application (Android, iOS, Javascript, Node.js, Java, Unity, PHP, C++ ...). Il propose d'héberger en NoSQL et en temps réel des bases de données, du contenu, de l'authentification sociale (Google, Facebook, Twitter et Github), et des notifications, ou encore des services, tel que par exemple un serveur de communication temps réel. Lancé en 2011 sous le nom d'Envolve, par Andrew Lee et par James Templin, le service est racheté par Google en octobre 2014. Il appartient aujourd'hui à la maison mère de Google : Alphabet. Toute l'implémentation et la gestion serveur de Firebase est à la charge exclusive de la société Alphabet. Les applications qui utilisent Firebase intègrent une bibliothèque qui permet les diverses interactions possibles.
\begin{figure}[!h]
\begin{center}
\includegraphics[width=6cm]{presentation image/Firebase.png}
\end{center}
%légende de l'image
\caption{logo Firebase}
\end{figure}


\section{Architecture de l'application}


• application : contient les services qui gèrent les abonnements aux flux de données d'infrastructure (pas directement, mais en utilisant le principe d'inversion de dépendance).\\

• config : contient la configuration de la base de données pour le stockage en local et une config des api Rest ainsi qu’une configuration de l’injection de dépendances en utilisant le package get-it.\\

• domain : contient seulement les entités internes de l’application. Ce module indépendant de tous les autres modules.\\

• infrastructure : contient les Data Source qui adaptent les réponses JSON obtenues d'un serveur aux DTO et vice-versa ainsi que les repositories qui prennent le rôle d’une frontière entre la couche de domain et d’application.\\

• presentation : contient les widgets qui constituent nos pages mobiles avec la gestion des états.\\

• utils : contient les helpers, services, UI utils, mixins qui sont utilisés dans toute l’application.\\

• analysis-options.yaml : ce fichier contient tous les options d’analyse utilisé par Flutter tools.\\

• pubspec.yaml : contient tous les dépendances utilisés dans notre projet moile.

\begin{figure}[!h]
\begin{center}
\includegraphics[width=6cm]{presentation image/Screenshot 2021-12-03 at 00.21.00.png}
\end{center}
%légende de l'image
\caption{Architecture des dossiers}
\end{figure}